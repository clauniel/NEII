\input{pre}
\tikzset{rrail/.style={rground,yscale=-1}}
\begin{document}
\input{frontpage}
\newpage
\setcounter{section}{1}
\begin{figure}
    \center
    \begin{subfigure}{0.3\textwidth}
        \center
        \begin{circuitikz} \draw
            (0,0) node[nmos] (n) {}
            (n.source) node[anchor=north] {S}
            (n.drain) node[anchor=south] {D}
            (n.gate) node[anchor=east] {G}
            (n.drain) to[short, i=$I_{ds}$] ++(0,-0.2)
        ;\end{circuitikz}
        \caption{}
    \end{subfigure}
    \begin{subfigure}{0.3\textwidth}
        \center
        \begin{circuitikz} \draw
            (0,0) node[pmos] (p) {}
            (p.source) node[anchor=south] {S}
            (p.drain) node[anchor=north] {D}
            (p.gate) node[anchor=east] {G}
            (p.source) to[short, i=$I_{ds}$] ++(0,-0.2)
        ;\end{circuitikz}
        \caption{}
    \end{subfigure}
    \caption{nFET (a) with gate (G), drain (D) and source (S), and a pFET (b) with the same terminals.}
    \label{fig:fets}
\end{figure}
\section{Transistor Operation in the Subthreshold Domain}
\subsection{(a) nFET drawing}
See Fig.~\ref{fig:fets}.
\subsection{(b) Subthreshold current expression}
The expression for the current from drain to source, \(I_{ds}\) in an NFET is 
\begin{equation*}
    I_{ds} = I_0e^{\frac{\kappa V_g}{U_T}}\left(e^{\frac{-V_s}{U_T}}-e^{\frac{-V_d}{U_T}}\right)
\end{equation*}
Which can be split up into the positive forward current 
\begin{equation*}
    I_f = I_0e^{\frac{\kappa V_g-V_s}{U_T}}
\end{equation*}
And negative reverse current
\begin{equation*}
    I_r = I_0e^{\frac{\kappa V_g - V_d}{U_T}}
\end{equation*}
\subsection{(c) Effects of drain and source voltages}
Decreasing the source voltage \(V_s\) has the effect of (i) increasing the forward current, (ii) but it has no effect on
the reverse current.

Conversely, decreasing the drain voltage \(V_d\) (i) does not affect the forward current, (ii) but it increases the reverse current.

\subsection{(d) Relative magnitudes of forward and reverse currents}
A transistor operating in subthreshold is in saturation when \(V_{ds}>4U_T\). Given that \(e^4 \simeq 55\), the relative magnitudes of 
the forward and reverse currents are (i) less than two orders of magnitude in the ohmic region and (ii) more than two orders of magnitude 
in saturation.

\subsection{(e) Approximate values of \(V_{ds}\)}
A transistor in subthreshold is in the ohmic region when \(V_{ds}<4U_T\) and in saturation when \(V_{ds}>4U_T\).

\subsection{(f) pFETs}
See Fig.~\ref{fig:fets} for the pFET drawing.

The expression for current through a pFET in subthreshold is
\begin{equation*}
    I_{ds} = I_0e^{\frac{\kappa (V_w-V_g)}{U_T}}\left(e^{\frac{(V_w-V_s}{U_T}}-e^{\frac{(V_w-V_d}{U_T}}\right)
\end{equation*}
Where \(V_w\) is the bulk voltage of the pFET transistor.

This can be broken up into a forward current
\begin{equation*}
    I_f = I_0e^{\frac{\kappa (V_w-V_g)-(V_w-V_s)}{U_T}}
\end{equation*}
And a reverse component
\begin{equation*}
    I_r = I_0e^{\frac{\kappa (V_w-V_g)-(V_w-V_d)}{U_T}}
\end{equation*}

\section{Linear Resistance Circuit}
\begin{figure}
    \center
    \begin{subfigure}{0.3\textwidth}
        \center
        \begin{circuitikz} \draw 
            (0,0) node[pmos, rotate=-90] (p) {}
            (p.B) node[anchor=north] {$M_1$}
            (p.source) node[anchor=west] {$V_2$}
            (p.drain) node[anchor=east] {$V_1$}
            (p.gate) node[anchor=south] {$V_b$}
            (p.drain) to[short,i=$I_{M1}$] ++(0.1,0)
        ;\end{circuitikz}
    \end{subfigure}
    \begin{subfigure}{0.3\textwidth}
        \center
        \begin{circuitikz} \draw
            (0,0) node[nmos] (m4) {}
            (m4.drain) to[short] ++(0,0.0) node[rrail] {}
            (m4.B) node[anchor=west] {$M_4$}
            (0,-2) node[nmos] (m3) {}
            (m3.B) node[anchor=west] {$M_3$}
            (m3.source) to[short, i=$I_b$] ++(0,-0.1) node[ground] {}
            (m3.drain) -- (m4.source) -- ++(0.5,0) node[anchor=west] {$V_b$}
            (m4.gate) node[anchor=east] {$V_1$}
            (m3.gate) node[anchor=east] {$V_m$}
        ;\end{circuitikz}
    \end{subfigure}
    \caption{Linear resistance circuit.}
    \label{fig:linres}
\end{figure}

\subsection{(a) Expression for current through M1 and M4}
The common mode voltage \(V_{cm}\) is the average of the inputs.
\begin{equation*}
    V_{cm} = \frac{V_1+V_2}{2}
\end{equation*}
And the differential voltage is 
\begin{equation*}
    \Delta V = V_1 - V_2
\end{equation*}
From this it follows that
\begin{equation*}
    V_1 = V_{cm} + \frac{\Delta V}{2} \quad , \quad V_2 = V_{cm} - \frac{\Delta V}{2}
\end{equation*}
And therefore the expression for the current through M1 can be rewritten as
\begin{equation}
    I_{M1} = I_0e^{\frac{\kappa (V_w - V_b) - (V_w - V_{cm})}{U_T}}\left(e^{\frac{\Delta V}{2U_T}}-e^{\frac{-\Delta V}{2U_T}}\right)
    \label{eq:IM1}
\end{equation}
For positive current in the direction noted in Fig.~\ref{fig:linres}.

The current through M4 is \(I_b\) and given by 
\begin{equation*}
    I_b = I_0e^{\frac{\kappa V_1 - V_b}{U_T}}
\end{equation*}
Assuming that M4 is in subthreshold saturation.

\subsection{(b) M1 current continued}
Substituting 
\begin{equation*}
    V_b = \kappa V_1 - U_T\ln\left(\frac{I_b}{I_0}\right) = \kappa \left( V_{cm} + \frac{\Delta V}{2} \right) - U_T\ln\left(\frac{I_b}{I_0}\right)
\end{equation*}
In~\ref{eq:IM1} yields
\begin{equation*}
    I_{M1} = I_0e^{\frac{\kappa (V_w - \kappa \left(V_{cm}+\frac{\Delta V}{2}\right) + U_T\ln\left(\frac{I_b}{I_0}\right)) - (V_w - V_{cm})}{U_T}}\left(e^{\frac{\Delta V}{2U_T}}-e^{\frac{-\Delta V}{2U_T}}\right)
\end{equation*}
Which reduces to
\begin{equation*}
    I_{M1} = I_b^\kappa e^{\frac{\kappa (V_w - \kappa \left(V_{cm}+\frac{\Delta V}{2}\right)) - (V_w - V_{cm})}{U_T}}\left(e^{\frac{\Delta V}{2U_T}}-e^{\frac{-\Delta V}{2U_T}}\right)
\end{equation*}
\begin{figure}
    \center
    \includegraphics{matlabtimes.eps}
    \caption{Resistive element current versus \(\Delta V\) in arbitrary units. The curve moves upwards for increasing \(I_b\)}
    \label{fig:absI}
\end{figure}

Fig.~\ref{fig:absI} shows how the current behaves as a function of \(\Delta V\). The slope is smaller for positive differential voltages
because the gate voltage of the pFET transistor in the resistive element depends on \(\Delta V\), and higher gate voltages result in 
lower currents.
\end{document}
