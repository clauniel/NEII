\input{pre}

\tikzset{rrail/.style={rground,yscale=-1}}
\newcommand{\reffig}[1]{Fig.~\ref{#1}}

\begin{document}
\input{frontpage}
\newpage
\setcounter{section}{1}
\section{Transistor as a Resistive Element: Conductance Linearity}
\begin{figure}
    \center
    \begin{subfigure}{0.3\textwidth}
        \center
        \begin{circuitikz}[american voltages] \draw 
            (0,0) node[pmos, rotate=-90] (p) {}
            (p.B) node[anchor=north] {$M_1$}
            (p.source) node[anchor=west] {$V_2$}
            (p.drain) node[anchor=east] {$V_1$}
            (p.gate) node[anchor=south] {$V_b$}
            (p.drain) to[short,i=$I_{M1}$] ++(0.1,0)
            (0,-3) node[ground] {}
            (0,-3) to[V=$V_{cm}$] (0,-2)
            (0,-2) -- (0,-2 -| p.drain) to[V=$\frac{V_{diff}}{2}$] (p.drain)
            (p.source) to[V=$\frac{V_{diff}}{2}$] (0,-2 -| p.source) -- (0,-2)
        ;\end{circuitikz}
    \end{subfigure}
    \begin{subfigure}{0.3\textwidth}
        \center
        \begin{circuitikz} \draw
            (0,0) node[nmos] (m4) {}
            (m4.drain) to[short] ++(0,0.0) node[rrail] {}
            (m4.B) node[anchor=west] {$M_4$}
            (0,-2) node[nmos] (m3) {}
            (m3.B) node[anchor=west] {$M_3$}
            (m3.source) to[short, i=$I_b$] ++(0,-0.1) node[ground] {}
            (m3.drain) -- (m4.source) -- ++(0.5,0) node[anchor=west] {$V_b$}
            (m4.gate) node[anchor=east] {$V_1$}
            (m3.gate) node[anchor=east] {$V_m$}
        ;\end{circuitikz}
    \end{subfigure}
    \caption{Linear resistance circuit with \(V_{diff}\) applied to the pFET drain and source.}
    \label{fig:linres}
\end{figure}
\begin{figure}
    \center
    \includegraphics{exp2.eps}
    \caption{Current through the resistive element with $\Delta V = V_1-V_2$, $V_{cm}=0.9$ V and $V_m=0.2$ V.}
    \label{fig:exp2}
\end{figure}
Fig.~\ref{fig:exp2} shows the current through the resistive element with fixed $V_m$ and $V_{cm}$. For positive \(\Delta V\),
the current follows the expression derived in the prelab
\begin{equation*}
    I_{M1} = I_b^\kappa e^{\frac{\kappa (V_w - \kappa \left(V_{cm}+\frac{\Delta V}{2}\right)) - (V_w - V_{cm})}{U_T}}\left(e^{\frac{\Delta V}{2U_T}}-e^{\frac{-\Delta V}{2U_T}}\right)
\end{equation*}
However, for negative \(\Delta V\), the pFET transistor moves out of subthreshold and the current is no longer linear in a log plot.
\section{Transistor as a Resistive Element: Varying Conductance}
\begin{figure}
    \center
    \includegraphics{exp3.eps}
    \caption{Current through the resistive element with $V_{cm}=0.9$ V and varying $V_m$.}
    \label{fig:exp3}
\end{figure}
The effect of varying $V_m$ can be seen in Fig.~\ref{fig:exp3}. The current through the resistive element increases, because $I_b$ increases
and pulls the gate voltage of the pFET down, which increases current. This also causes the pFET to go above threshold more quickly for 
negative \(\Delta V\).
\section{Transistor as a Resistive Element: Common Mode Sensitivity}
\begin{figure}
    \center
    \includegraphics{exp4.eps}
    \caption{Current through the resistive element with $V_m=0.2$ V and varying $V_{cm}$.}
    \label{fig:exp4}
\end{figure}
Fig.~\ref{fig:exp4} shows the effect of changing the common mode voltage. Since the common mode voltage is coupled to the gate voltage of the pFET, the current will go down. 
However, increasing the common mode voltage also increases the drain and source voltages. Since these are directly coupled to the final current, and not by a factor of
\(\kappa\) like the gate voltage, the net effect is an increase in current.
\section{Transistor as a Resistive Element: Symmeterizing the Circuit}
\begin{figure}
    \center
    \includegraphics{exp53.eps}
    \caption{Current through the symmeterized resistive element with $V_{cm} = 0.9$ V and varying $V_m$.}
    \label{fig:exp53}
\end{figure}
\begin{figure}
    \center
    \includegraphics{exp54.eps}
    \caption{Current through the symmeterized resistive element with $V_m = 0.2$ V and varying $V_{cm}$.}
    \label{fig:exp54}
\end{figure}
The repeated experiments for the symmeterized resistive elements are shown in Figs.~\ref{fig:exp53} and~\ref{fig:exp54}. 
They are exactly the same as for the initial one, except the current is mirrored around the y-axis. This is because the
circuit essentially is the same, only with \(\Delta V\) and the direction of current being reversed.
\end{document}
