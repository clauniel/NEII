\section{Discussion and Conclusion}
We attempted to create a pixel that would produce 12 ON and 12 OFF for a change in contrast by a factor 3.3, regardless
of transistor mismatch within the limits our process. The post-layout analysis revealed a parasitic capacitance
which decimated the high capacitive gain in the differencing circuit that we relied on to achieve extremely low
variance in the number of output spikes for a given input. It is possible that the effect of this parasitic capacitance
coul be eliminated by using an even higher gain. However, we did not investigate this.

Our post-layout pixel has a very low fill factor of 1.64\%, a high power consumption of 690.4 nW on average, a
relatively low bandwidth of 21.4 kHz, and its length of 506 \(\mu\)m makes it extremely long. 
These factors were sacrificed in the attempt to obtain zero variance in the number of events generated by the 
test input.

The pre-layout circuit showed that with a high enough gain, it is possible to achieve zero variance, making our
pre-layout FOM infinitely high. However, the real FOM obtained post-layout was \(1.27\cdot10^7\). 
In light of this, it would likely have been better to try and find a local optimal tradeoff between length, power and
mismatch sensitivity.
