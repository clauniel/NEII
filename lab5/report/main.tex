\input{pre}

\tikzset{rrail/.style={rground,yscale=-1}}
\newcommand{\reffig}[1]{Fig.~\ref{#1}}

\begin{document}
\input{frontpage}
\newpage
\section{Demux Layout}
The Demux circuit performs the function of an active low one-hot decoder and performs the function of the
truth table in Table~\ref{tab:demux}.
\begin{table}
    \center
    \begin{tabular}{c  c | c c c}
        \texttt{IN0} & \texttt{IN1} & \texttt{nsel0} & \texttt{nsel1} & \texttt{nsel2} \\ \hline
        0 & 0 & 1 & 1 & 1 \\
        0 & 1 & 0 & 1 & 1 \\
        1 & 0 & 1 & 0 & 1 \\
        1 & 1 & 1 & 1 & 0 \\
    \end{tabular}
    \caption{Demux truth table}
    \label{tab:demux}
\end{table}
From the truth table we can read the boolean equations
\begin{align*}
    \texttt{nsel0} & = \texttt{IN0} +  \overline{\texttt{IN1}} \\
    \texttt{nsel1} & = \overline{\texttt{IN0}} +  \texttt{IN1} \\
    \texttt{nsel2} & = \overline{\texttt{IN0}} +  \overline{\texttt{IN1}} 
\end{align*}
Which can be transformed to their NAND form for a smaller circuit
\begin{align*}
    \texttt{nsel0} & = \overline{\overline{\texttt{IN0}} \cdot  \texttt{IN1}} \\
    \texttt{nsel1} & =\overline{\texttt{IN0} \cdot  \overline{\texttt{IN1}}} \\
    \texttt{nsel2} & = \overline{\texttt{IN0} \cdot \texttt{IN1}}
\end{align*}
\begin{figure}
    \center
    \includegraphics{demux_layout.png}
    \caption{Layout of the demux cell.}
    \label{fig:demux_layout}
\end{figure}
We constructed the circuit using the standard logic cells. The final layout can be seen in Fig.~\ref{fig:demux_layout}.

\section{Chip Core}
The chip core contains analog buffers that buffer the output of the follower integrators. We need them because connecting
the unbuffered outputs of the follower integrators directly to the bonding pads introduces a large capacitance which they
cannot drive.

Fig.~\ref{fig:cell_layout} shows the layout of the chip core in Fig.~5.1 of the lab manual. The digital and analog grounds
are separated because the digital ground moves when the clock signal switches. Using the same ground for analog and digital
circuits would disturb the characteristics of the analog circuits.
\begin{figure}
    \center
    \includegraphics[angle=90,width=\textwidth]{cell_layout.png}
    \caption{Layout of circuit from Fig. 5.1 in the lab manual.}
    \label{fig:cell_layout}
\end{figure}

\end{document}
