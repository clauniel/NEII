\input{pre}

\tikzset{rrail/.style={rground,yscale=-1}}
\newcommand{\reffig}[1]{Fig.~\ref{#1}}

\begin{document}
\input{frontpage}
\newpage
\section{Post-Lab}
We used square transistors of width and length 1.2\(\mu\)m when constructing the follower integrator. With a gate oxide
of thickness 4.58nm as calculated in lab 3, the mismatch voltage between matching transistors is 
\begin{equation*}
    \sigma = \frac{4.58\text{mV}}{1.2} = 3.82 \text{mV}
\end{equation*}
Ignoring the bias transistors and assuming that the mismatch between follower integrators is independent, the delay line
consists of a chain of 8 matched transistors, two pairs in every follower integrator and one pair between the 1st to 2nd and 2nd to 3rd integrator. This gives a total standard deviation of 
\begin{equation*}
    \sigma_T = \sqrt{8\cdot\sigma^2} = 10.80 \text{mV}
\end{equation*}
This figure also ignores the systematic offset between follower integrators, caused by the non-infinite gain which slightly
drops the output voltage in relation to the input voltage even for DC signals.

The mismatch in the bias transistors should not affect the DC offset on the output as the bias current affects the time constant
and therefore temporal dynamics of the circuit. A mismatch in the bias transistor would change the corner frequency for the 
individual filters, but not the DC offset.
\end{document}
