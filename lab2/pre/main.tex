\input{pre}
\tikzset{rrail/.style={rground,yscale=-1}}
\begin{document}
\input{frontpage}
\newpage
\section{The Axon-Hillock Circuit}
The axon-hillock circuit needs a current to be supplied to the membrane node. Assuming that the initial conditions have the
output at ground, supplying a current will start charging the membrane capacitor. Once the voltage on the membrane 
capacitor reaches the tipping point of the first inverter, it will flip and cause the second inverter to send the
output to $V_{dd}$. The chance in the output is coupled back to the membrane potential which will increase by 
\begin{equation*}
    \Delta V_{mem} = \frac{C_{f}}{C_f+C_{mem}} V_{dd}
\end{equation*}
Where $C_f$ is the feedback capacitance and $C_{mem}$ the membrane capacitance.

When the output goes high, the membrane capacitor will start to discharge through the now open transistor providing 
a path to ground. Once the membrance potential goes below the first inverters tipping point, the reverse action will 
occur and the positive feedback through the feedback capacitor will decrease the membrance node voltage by 
\begin{equation*}
    \Delta V_{mem} = \frac{-C_{f}}{C_f+C_{mem}} V_{dd}
\end{equation*}
We can calculate the amount of time the output spends being high and low in steady state for a given input current. These
are, respectively
\begin{equation*}
    T_H = \frac{(C_f+C_{mem})\Delta V_{mem}}{I_r-I_{in}} = \frac{C_fV_{dd}}{I_r-I_{in}}
\end{equation*}
and
\begin{equation*}
    T_L = \frac{(C_f+C_{mem})\Delta V_{mem}}{I_{in}} = \frac{C_fV_{dd}}{I_{in}}
\end{equation*}
Because the charge that must be removed from the membrane node is given by the capacitance of the node and the change in 
voltage. 
\end{document}
